%%%%%%%%%%%%%%%%%%%%%%%%%%%%%%%%%%%%%%%%%%%%%%%%%%%%%%%%%%%%%%%%%%%
%     LATEX TEMPLATE                               %
%%%%%%%%%%%%%%%%%%%%%%%%%%%%%%%%%%%%%%%%%%%%%%%%%%%%%%%%%%%%%%%%%%%
%
\documentclass[12pt,a4paper]{article}

\usepackage[T1]{fontenc}                % Verarbeitung von Umlauten
\usepackage[utf8]{inputenc}		      	% Eingabe von Umlauten
\usepackage[ngerman,USenglish]{babel}             % Rechtschreibung und Trennung
\usepackage{amssymb,amsmath,amsfonts}   % F�r mathematische Darstellungen
\usepackage{setspace}\onehalfspacing    % Zeilenabstand
\usepackage[margin=3cm,includefoot]{geometry}    % Rand
\usepackage{natbib}
\usepackage{graphicx}

%\begin{document}

%---------- Define variables for cover sheet ----------%

% Angaben zum Deckblatt: Bitte vervollst�ndigen
\def\dbtitle    	{BlA BLA BLA }   					% z.B. Competition and Efficiency
\def\dbsupervisor 	{Me}        % e.g. Karl Marx
\def\dbname     	{Christoph Winter}                  	% e.g. Friedrich Engels
\def\dbmajor    	{Econ PhD}           			% e.g. VWL (Magister)
\def\dbsem		 	{25}          		% e.g. 7
\def\dbadrs     	{Address line 1}     		% e.g. Geschwister-Scholl-Platz 1
\def\dbadrt     	{Address line 2}      		% e.g. 01259 M�nchen

% Semester und Titel des Schwerpunktseminars: Eventl. anpassen
\def\dbsemester{Summer Term 2016}
\def\dbsemtitle{Topics in International Migration}


\newcommand{\test}{\dbsemtitle}

%----- Cover Sheet -----%
\newcommand{\makedeckblatt}{
 \begin{titlepage}
   \vspace*{1.5cm}
   \begin{center}
 	\Huge
     \dbtitle\\
     \vspace{3.5cm}
     \Large
     \textsc{Schwerpunktseminar\\
     \dbsemester}\\
     \vspace{0.5cm}
 	\large
     \dbsemtitle
   \end{center}
   \normalsize
   \vfill
   \begin{tabular*}{\linewidth}{l@{\extracolsep\fill}l}
 	Submitted by:     		&  Supervisor:   \\
 	\dbname              	&  \dbsupervisor \\
 	\dbmajor              		\\
 	\dbsem. semester 		\\ [1ex]
 	\dbadrs              		\\
 	\dbadrt
   \end{tabular*}
 \end{titlepage}
}

\newcommand{\dbandtoc}{
  \makedeckblatt
  \pagenumbering{roman}
  \tableofcontents
  \newpage
 % \addcontentsline{toc}{section}{\listfigurename} %% comment this line if you do not want the list of figures in your table of content
  %\listoffigures %% You do not need a list of figures for the seminar paper
  \newpage
  %\addcontentsline{toc}{section}{\listtablename} %% same here
  %\listoftables %% You do not need a list of tables for the seminar paper
 % \thispagestyle{empty}
  \pagebreak
  \pagenumbering{arabic}
}



\begin{document}

% Table of Contents and Cover Sheet (automatically) 
\dbandtoc

%%%%%%%%%%%%%%%%%%%%%%%%%%%%%%%%%%%%%%%%%
%           Start of Thesis          %
%%%%%%%%%%%%%%%%%%%%%%%%%%%%%%%%%%%%%%%%%

 \parindent 0pt 		% Have no indent on the first line of each paragrap

\section{Introduction}

ABC\footnote{This is a footnote.}

 
\section{This might be useful}

\subsection{Make Comments}
% This is a comment that doesn't show up

\section{Useful Commands}

\textbf{This is bold text.} This not.\footnote{This is the second footnote.}
In diesem ist das letzte Wort \textit{kursiv}.

New line.
No new line \\
\\
New line

New Paragraph

\begin{small}
This is small.
This is also small.
\begin{tiny}
Very tiny. 
\end{tiny}
\end{small}

\begin{tiny}
Very tiny. 
\end{tiny}

\subsection{Tables and Figures}
\label{subsec:TablesFigures}

\subsection{Equations}

$x=1$ or $x^{2+54}=y_{2+3} \sum_{n=1}^{N} \beta \rho \Delta$

\begin{equation}
y_{2+3} \sum_{n=1}^{N} 
\end{equation}




\subsubsection{Make own Table}

\begin{table}[h]
\caption{This is my first Table}
\label{tab:WageChange}

\begin{center}
\begin{small}

\begin{tabular}{l c r}

\hline \hline
\textbf{City 1} & \textbf{City 2} & \textbf{City 3} \\
\hline
Hamburg & München & Dortmund \\
& & \\
Ende1 & Ende2 & Ende 3 \\
\hline
\end{tabular}
\end{small}
\end{center}

\end{table}

\subsection{Include Figures}

\begin{figure}[h]
\caption{ABC}
\center
\includegraphics[scale=0.75]{e5_raiseabs_hist}
\includegraphics[scale=0.75]{density_bmi}

\end{figure}

\begin{figure}[h]
\caption{ABC}
\center
\includegraphics[scale=0.40]{e5_raiseabs_hist}
\includegraphics[scale=0.40]{density_bmi}
\end{figure}

\subsection{References to Tables and Figures}

Like the authors show in Table\ref{tab:WageChange} .... 


Like we have already discussed in Subsection \ref{subsec:TablesFigures}
\section{Discussion}



\subsection{Pros}
This are the pros.
\subsubsection{Pro1}
Pro 1.....
This idea was developed by \cite{borjas1994economics}. Migration is bad  \citep{borjas1994economics}
\subsubsection{Pro2}
Pro 2....
This is where you put the content!!
\subsection{Cons}
This are the cons.


%Not sure about that 
%Borjas (2008) shows that XY reduces migration.

\clearpage

\bibliographystyle{econometrica}
\bibliography{references}


\end{document}
